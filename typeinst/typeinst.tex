
\documentclass[runningheads,a4paper]{llncs}

\usepackage{amssymb}
\setcounter{tocdepth}{3}
\usepackage{graphicx}

\usepackage{url}
\urldef{\mailsa}\path|{alfred.hofmann, ursula.barth, ingrid.haas, frank.holzwarth,|
\urldef{\mailsb}\path|anna.kramer, leonie.kunz, christine.reiss, nicole.sator,|
\urldef{\mailsc}\path|erika.siebert-cole, peter.strasser, lncs}@springer.com|    
\newcommand{\keywords}[1]{\par\addvspace\baselineskip
\noindent\keywordname\enspace\ignorespaces#1}

\begin{document}

\mainmatter  % start of an individual contribution

% first the title is needed
\title{Das \\ Flussproblem}

% a short form should be given in case it is too long for the running head
\titlerunning{Das \\ Flussproblem}

% the name(s) of the author(s) follow(s) next
%
% NB: Chinese authors should write their first names(s) in front of
% their surnames. This ensures that the names appear correctly in
% the running heads and the author index.
%
\author{Jan Niklas Hollenbeck \\ und \\ Marco Leeske}
\institute{Hochschule Darmstadt}
\maketitle


\begin{abstract}
In diesem Paper wird das Flussproblem beleuchtet, welches ein mathematisches Problem zur Findung des maximalen Flusses in Netz\-werken beschreibt.
 Probleme des realen Lebens, beispielsweise in Kanal- oder Verkehrsleitsystemen, werden als gerichtete Graphen mo\-delliert und mittels Algorithmen gelöst.
 Zur Lösung des Flussproblems gibt es unterschiedliche Algorithmen, welche sich in Laufzeit und Anwendungsfall unterscheiden.
 Die vorliegende Arbeit soll einen Überblick über zwei verschiedene Algorithmen sowie deren Anwendungsbereiche geben, um die Auswahl, des optimalen Algorithmus für den jeweils vorliegenden Anwendungsfall, zu erleichtern.
 Es werden die Funktionsweisen der Algorithmen von Ford und Fulkerson sowie Edmond und Karp erläutert, empirisch untersucht und es wird ein Laufzeitvergleich durchgeführt.
 Dies wird mit Hilfe eines Java Programmes, welches für jeden Datensatz alle Algorithmen testet, realisiert.
Es werden die jeweiligen Vor- und Nachteile der Algorithmen aufgezeigt, der Funktionsumfang geprüft und anschließend die gesammelten Resultate der Laufzeittests verglichen, um sowohl eine Übersicht als auch eine Entscheidungshilfe geben zu können.
 Die Frage, welcher Algorithmus bei unterschiedlichen Ausgangssituationen und Erwartungen den Vorzug erhält, wird sich nach der Lektüre dieser Arbeit beantworten lassen. 

\end{abstract}



\end{document}
